\documentclass[a4paper]{article}
\usepackage[margin=2cm]{geometry}

\usepackage{xgreek}
\usepackage{xltxtra}
\usepackage{unicode-math}
\usepackage{graphicx}
\usepackage{float}
%\usepackage{tocbibind}
\usepackage[pdfusetitle]{hyperref}
\graphicspath{{./images/}}
\setmainfont[Mapping=tex-text]{CMU Serif}
\setsansfont[Mapping=tex-text]{CMU Sans Serif}
\setmonofont{CMU Typewriter Text}
\usepackage{minted}

\usepackage{tocloft}
\cftsetindents{section}{0em}{3em}

\title{7η Εργαστηριακή Άσκηση\\Εργαστήριο Μικροϋπολογιστών}
\author{Ανδρέας Στάμος (03120018), Κωνσταντίνος Πίκουλας (03120112)}
\date{Δεκέμβριος 2023}

\hypersetup{
	colorlinks=true,
	linkcolor=blue,
}

\begin{document}


\maketitle
\tableofcontents

\section{Ζήτημα 1 -- Ανάπτυξη συναρτήσεων επικοινωνίας 1-Wire με το θερμόμετρο}

\par Αρχικά εκτελούμε \texttt{one\_wire\_reset} προκειμένου να θέσουμε το θερμόμετρο σε κατάσταση λειτουργίας.
Έπειτα στέλνουμε εντολή (\texttt{0xCC})πως απευθυνόμαστε στην μοναδική συνδεδεμένη συσκεύη (παρακάμπτωντας την αποστολή διεύθυνσης έτσι),
και έπειτα στέλνουμε εντολή θερμομέτρησης (\texttt{0x44}). Μέχρι να ολοκληρωθεί η θερμομέτρηση, το θερμόμετρο μας στέλνει 0, αν ζητήσουμε να
στείλει ένα bit (με την \texttt{one\_wire\_receive\_bit}). Μόλις μας στείλει 1, η θερμομέτρηση έχει ολοκληρωθεί.
\par Έπειτα θέτουμε ξανά την συσκεύη σε κατάσταση λειτουργίας εκ νέου με \texttt{one\_wire\_reset}, την επιλέγουμε στέλνοντας \texttt{0xCC}, και στέλνουμε
εντολή ανάγνωσης της μνήμης της θερμομέτρου (\texttt{0xBE}). Τέλος, διαβάζουμε 2 bytes (με \texttt{one\_wire\_receive\_byte}), πρώτα το LSB και έπειτα
το MSB (little-endian) και τα ενώνουμε σε μία προσημασμένη 16bit μεταβλητή (int16\_t).

\par Ο κώδικας έχει διαχωριστεί σε χωριστά αρχεία, με σκοπό να είναι modular.

\par Ο κώδικας για την επικοινωνία με το θερμόμετρο βρίσκεται στο αρχείο \texttt{thermometer.c} που υπάρχει στο παράρτημα \ref{onewire_c},
με την αντίστοιχη C επικεφαλίδα να βρίσκεται στο παράρτημα \ref{thermometer_h}.

\par Ο κώδικας για την One Wire επικοινωνία -- χρησιμοποιείται από το \texttt{thermometer.c}
βρίσκεται στο αρχείο \texttt{onewire.c} που υπάρχει στο παράρτημα \ref{onewire_c},
με την αντίστοιχη C επικεφαλίδα να βρίσκεται στο παράρτημα \ref{onewire_h}.

\section{Ζήτημα 2 -- Απεικόνιση θερμοκρασίας στην LCD οθόνη}

\par Ο κώδικας για την επικοινωνία με την LCD οθόνη συντάχθηκε σε προηγούμενη εργασία και παρατίθεται εδώ ξανά. Πιο συγκεκριμένα:
\begin{itemize}
	\item \texttt{lcd.c} στο παράρτημα \ref{lcd_c} και \texttt{lcd.h} στο παράρτημα \ref{lcd_h} -- για την επικοινωνία με την οθόνη.
	Πρέπει κατά την μεταγλώττιση του \texttt{lcd.c} να γίνει \texttt{define} η macro \texttt{LCD\_PORTD} ή \texttt{LCD\_PCA9555} ανάλογα
	το πώς επιθυμούμε να γίνεται η επικοινωνία. Εδώ επιθυμούμε μέσω PCA9555 διότι το PD4 χρησιμοποιείται για την 1-wire επικοινωνία με το θερμόμετρο.
	\item \texttt{twi\_pca9555.c} στο παράρτημα \ref{twi_pca9555_c} και \texttt{twi\_pca9555.h} στο παράρτημα \ref{twi_pca9555_h}
	για την επικοινωνία TWI και επιπρόσθετα για την επικοινωνία μέσω TWI με το ολοκληρωμένο PCA9555.
\end{itemize}

\par Αρχικά διαβάζουμε την θερμοκρασία από το θερμόμετρο με την συνάρτηση που αναπτύχθηκε στο Ζήτημα 1. Αν δεν ανιχνευθεί συσκευή, εκτυπώνουμε \texttt{NO Device}
στην οθόνη. Διαφορετικά εκτυπώνουμε την θερμοκρασία με 3 δεκαδικά ψηφία. Αρχικά μετατρέπουμε τον fixed-point αριθμό που μας δίνει το θερμόμετρο (4 bits κλασματικό
μέρος και 8 bits ακέραιο), σε αριθμό κινητής υποδιαστολής τύπου \texttt{float} και στην συνέχεια με χρήση της συνάρτησης \texttt{snprintf} μετατρέπουμε τον αριθμό
αυτό σε συμβολοσειρά. Τέλος εκτυπώνουμε την συμβολοσειρά στην οθόνη. Δόθηκε έμφαση στην σωστή ερμηνείας των αρνητικών αριθμών (δίνονται από το θερμόμετρο ως
συμπλήρωμα ως προς 2). Τέλος, προκειμένου η οθόνη να μην κάνει ``blink'', ανανενώνουμε το μήνυμα στην οθόνη μόνο αν η ανάγνωση θερμοκρασίας επιστρέψει καινούρια
τιμή σε σχέση με την προηγούμενη φορά που εκτυπώθηκε θερμοκρασία στην οθόνη. Αξίζει, επίσης να επισημανθεί, πως για να λειτουργήσει η snprintf με float,
πρέπει να γίνει link με την σωστή snprintf (η default δεν περιέχει λειτουργικότητα αριθμών κινητής υποδιαστολής για οικονομία χώρου), και αυτό φαίνεται παρακάτω
στο \texttt{Makefile}.

\par Ο κώδικας του ζητήματος είναι ο εξής:
\inputminted[breaklines, linenos]{c}{../Lab7.2/main.c}

\pagebreak

\par Τέλος, επειδή η μεταγλώττιση και η εγκατάσταση στην πλακέτα γίνεται χωρίς το MPLAB X, αλλά με μεταγλώττιση με AVR-GCC και εγκατάσταση με avrdude στην
μνήμη προγράμματος παρατίθεται το σχετικό \texttt{Makefile}:
\par Ο φάκελος \texttt{pack} που αναφέρεται είναι οι σχετικές βιβλιοθήκες με τις αντίστοιχες επικεφαλίδες
για τον ATmega328PB, καθώς δεν υποστηρίζεται από default από τον avr-gcc. Αρκεί κανείς να κατεβάσει από την σχετική ιστοσελίδα της Microchip
\url{https://packs.download.microchip.com/} το σχετικό pack για την ATmega Series και να το αποσυμπιέσει μέσα στον φάκελο \texttt{pack}.

\inputminted[breaklines, linenos]{basemake}{../Lab7.2/Makefile}

\pagebreak

\appendix

\section{Αρχείο thermometer.h}
\label{thermometer_h}
\inputminted[breaklines, linenos]{c}{../common/thermometer.h}

\section{Αρχείο thermometer.c}
\label{thermometer_c}
\inputminted[breaklines, linenos]{c}{../common/thermometer.c}

\section{Αρχείο onewire.h}
\label{onewire_h}
\inputminted[breaklines, linenos]{c}{../common/onewire.h}

\section{Αρχείο onewire.c}
\label{onewire_c}
\inputminted[breaklines, linenos]{c}{../common/onewire.c}

\section{Αρχείο lcd.h}
\label{lcd_h}
\inputminted[breaklines, linenos]{c}{../common/lcd.h}

\section{Αρχείο lcd.c}
\label{lcd_c}
\inputminted[breaklines, linenos]{c}{../common/lcd.c}

\section{Αρχείο twi\_pca9555.h}
\label{twi_pca9555_h}
\inputminted[breaklines, linenos]{c}{../common/twi_pca9555.h}

\section{Αρχείο twi\_pca9555.c}
\label{twi_pca9555_c}
\inputminted[breaklines, linenos]{c}{../common/twi_pca9555.c}

\end{document}

