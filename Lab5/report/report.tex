\documentclass[a4paper]{article}
\usepackage[margin=2cm]{geometry}

\usepackage{xgreek}
\usepackage{xltxtra}
\usepackage{unicode-math}
\usepackage{graphicx}
\usepackage{float}
%\usepackage{tocbibind}
\usepackage[pdfusetitle]{hyperref}
\graphicspath{{./images/}}
\setmainfont[Mapping=tex-text]{CMU Serif}
\setsansfont[Mapping=tex-text]{CMU Sans Serif}
\setmonofont{CMU Typewriter Text}
\usepackage{minted}

\title{5η Εργαστηριακή Άσκηση\\Εργαστήριο Μικροϋπολογιστών}
\author{Ανδρέας Στάμος (03120018), Κωνσταντίνος Πίκουλας (03120112)}
\date{Νοέμβριος 2023}

\hypersetup{
	colorlinks=true,
	linkcolor=blue,
}

\begin{document}


\maketitle
\tableofcontents

\section{Ζήτημα 5.1}

\par Το \texttt{IO\_PORT 0} ρυθμίζεται ως έξοδος. Μέσα σε έναν βρόχο, διαβάζουμε τα $A$, $B$, $C$, $D$ από το \texttt{PIND},
υπολογίζουμε την ζητούμενη λογική συνάρτηση και στέλνουμε το αποτέλεσμα στο PCA9555 στο \texttt{IO\_PORT 0}.

\par Οι κοινές συναρτήσεις για το TWI και την επικοινωνία με το PCA9555 βρίσκονται στο Παράρτημα \ref{twi_pca9555}.

\par Ο κώδικας C είναι ο παρακάτω:
\inputminted[breaklines, linenos]{c}{../Lab5.1/main.c}

\section{Ζήτημα 5.2}

\par Χρησιμοποιούμε την γραμμή \texttt{IO1\_0} ως γείωση (γράφοντας \texttt{0}) σε αυτή. Έτσι όταν ο χρήστης πατήσει κάποιο κουμπί,
η αντίστοιχο γραμμή εισόδου γειώνεται, όπως την διαβάζουμε ως \texttt{0}. Επειδή τα bits, τα δείχνουμε με την ίδια σειρά στο LEDs,
μπορούμε απλά να κάνουμε shift 4 θέσεις δεξιά αυτό που διαβάσαμε. Επίσης απαιτείται να υπολογίσουμε την λογική άρνηση αυτού που διαβάσαμε,
ώστε να μεταβούμε από την αρνητική λογική των κουμπιών όταν τα διαβάζουμε, στην θετική λογική των LEDs.

\par Οι κοινές συναρτήσεις για το TWI και την επικοινωνία με το PCA9555 βρίσκονται στο Παράρτημα \ref{twi_pca9555}.

\par Ο κώδικας C είναι ο παρακάτω:
\inputminted[breaklines, linenos]{c}{../Lab5.2/main.c}

\section{Ζήτημα 5.3}
\par Χρησιμοποιήσαμε τους κώδικες της 4ης εργαστηριακής σειράς, αλλάζοντας την εγγραφή στο PORTD σε εγγραφή στο \texttt{IO PORT 1}.
Προκειμένου να γλιτώσουμε την ανάγνωση που συνέβαινε για να διατηρούμε πάνω στο PORTD αν πρόκειται για command ή data, περνάμε το
αν πρόκειται για command ή data ως όρισμα στις αντίστοιχες συναρτήσεις.

\par Επίσης, προκειμένου να κάνουμε τις αλλαγές γραμμής, χρησιμοποιήσαμε την εντολή του μικροελεγκτή ST7066U της LCD οθόνης:
\texttt{Set DDRAM Address}. Για την 1η γραμμή, θέτουμε την διεύθυνση DDRAM σε 0x00 και για την 2η γραμμή θέτουμε την διεύθυνση
DDRAM σε 0x40 (σύμφωνα με το datasheet).


\par Οι συναρτήσεις για την επικοινωνία με την LCD οθόνη μέσω του PCA9555 βρίσκονται στο Παράρτημα \ref{lcd}.
\par Οι κοινές συναρτήσεις για το TWI και την επικοινωνία με το PCA9555 βρίσκονται στο Παράρτημα \ref{twi_pca9555}.

\par Ο κώδικας C είναι ο παρακάτω:
\inputminted[breaklines, linenos]{c}{../Lab5.3/main.c}

\appendix

\section{Κοινές συναρτήσεις για το TWI και την επικοινωνία με το PCA9555}
\label{twi_pca9555}

Το αρχείο \texttt{twi\_pca9555\_.c} που χρησιμοποιείται στους παραπάνω κώδικες, για το TWI και την επικοινωνία με το PCA9555 είναι το εξής:
\inputminted[breaklines, linenos]{c}{../common/twi_pca9555.c}

\section{Κοινές συναρτήσεις για την επικοινωνία με την LCD οθόνη μέσω PCA9555}
\label{lcd}

Το αρχείο \texttt{lcd.c} που χρησιμοποιείται στους παραπάνω κώδικες, για την επικοινωνία με την LCD οθόνη μέσω του PCA9555, είναι το εξής:
\inputminted[breaklines, linenos]{c}{../common/lcd.c}

\end{document}

